\chapter{Testing}

{ \Large\itshape
	\noindent Demonstrative materials.
}

\section{Code blocks}

%\input{src/book/99-testing/_file}

\newtcbinputlisting[use counter=monokaicode]{\monokaicodeinput}[4][]{%
	enhanced,
	noparskip,
	breakable,
	colback=monokai-code-bg,
	opacitybacktitle=.2,%
	fonttitle=\bfseries,
	colupper=white,
	enlarge top by=\baselineskip,
	enlarge bottom by=\baselineskip,
	before upper={%
		\lstdefinestyle{#2-methods}{%
			language=monokai-#2,
			keywords=[5]{#4},
			keywordstyle=[5]\color{monokai-green},
			#1,
		}%
	},
	bottomsep at break=10pt,
	topsep at break=20pt,
	listing only,
	listing options={%
		%
	},
	listing style = #2-methods,
	after upper={},
	top=-1mm,
	left=10mm,
	bottom=1mm,
	frame hidden,
	arc=4pt,
	outer arc=0pt,
	boxrule=0pt,
	frame code={%
%		\draw[red,line width=4mm,rounded corners] (frame.north west) rectangle (frame.south east);
	},
	listing file = {#3},
	underlay unbroken and first={%
		\path[
			draw=language-#2,
			line width = 1mm,
			rounded corners=\codeborderradius
		]
		([shift={(.33\codeborderwidth, .33\codeborderwidth)}]frame.south west)
			-- ([shift={(.33\codeborderwidth, -.33\codeborderwidth)}]frame.north west)
			-- ([shift={(-.33\codeborderwidth, -.33\codeborderwidth)}]frame.north east)
			-- ([shift={(-.33\codeborderwidth, .33\codeborderwidth)}]frame.south east)
			-- cycle;
	},
	overlay first={%
		\path[draw=language-#2, line width = 1mm]
			([shift={(.33\codeborderwidth, .33\codeborderwidth)}]frame.south west)
			[rounded corners=\codeborderradius]-- ([shift={(.33\codeborderwidth, -.33\codeborderwidth)}]frame.north west)
			[rounded corners=\codeborderradius]-- ([shift={(-.33\codeborderwidth, -.33\codeborderwidth)}]frame.north east)
			[sharp corners]-- ([shift={(-.33\codeborderwidth, .33\codeborderwidth)}]frame.south east)
			[sharp corners]-- cycle;
	},
	overlay last={%
		\path[draw=language-#2, line width = 1mm]
			([shift={(.33\codeborderwidth, .33\codeborderwidth)}]frame.south west)
			[sharp corners]-- ([shift={(.33\codeborderwidth, -.33\codeborderwidth)}]frame.north west)
			[sharp corners]-- ([shift={(-.33\codeborderwidth, -.33\codeborderwidth)}]frame.north east)
			[rounded corners=\codeborderradius]-- ([shift={(-.33\codeborderwidth, .33\codeborderwidth)}]frame.south east)
			[rounded corners=\codeborderradius]-- cycle;
	},
}

%\null

\noindent Syntactic sugar for relative file paths:

\monokaicodeinput{typescript}{../code/test-ts-syntax.ts}{%
	fn,
}

\noindent Inspecting particular lines within a code file:

\monokaicodeinput[%
linerange={7-19},
firstnumber=7,
]{python}{../code/factory.py}{%
	get_video_exporter,
	get_audio_exporter,
}



\makeatletter
\renewcommand{\readarray}[3][-]{\read@array[#1]{#2}{#3}}%

\pgfkeys{%
	/inputcodefile/args/.cd,
	filepath/.initial={},
	language/.initial={python},
	functions/.store in=\monokaihlfuncs,
}
%
\def\inputcodefile@set@keys#1{%
	\pgfkeys{%
		/inputcodefile/args/.cd,
		#1,
	}
}
\def\inputcodefile@get#1{%
	\pgfkeysvalueof{/inputcodefile/args/#1}%
}

\newcommand\monokaicode[2][]{%
	\bgroup%
		\inputcodefile@set@keys{#2}%
		\monokaicodeinput[#1]%
		{\inputcodefile@get{language}}%
		{\inputcodefile@get{filepath}}%
		{\monokaihlfuncs}%
	\egroup%
}
\makeatother

%\monokaicode[%
%	linerange={7-19},
%	firstnumber={7},
%]{%
%	filepath={../code/factory.py},
%	language={python},
%	functions={%
%		get_video_exporter,
%		get_audio_exporter,
%	},
%}









